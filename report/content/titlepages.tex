\pdfbookmark[0]{English title page}{label:titlepage_en}
\aautitlepage{%
  \englishprojectinfo{
  1 %Semester nr.
  }{
    Arduino BLE as presence sensor for statistical analysis %title
  }{%
    Optimizing Space Utilization with BLE Sensor Technology %theme
  }{%
    Fall Semester 2024 %project period
  }{%
    9 % project group
  }{%
    %list of group members
    Alexander B. Pedersen \\
    Alexander Hagelberg \\
    Andrej P. Maricic \\
    Janus L. Pedersen \\
    Jens L. H. Jaedeke \\
    Johannes D. Rørdam \\
    Mass M. Vedel
  }{%
    %list of supervisors
    Reza Tadayoni
  }{%
    \today % date of completion
  }%
}{%department and address
    \textbf{Aalborg University Copenhagen} \\
    A.C. Meyers Vænge 15\\
    2450 København SV \\
    \href{http://www.aau.dk}{http://www.aau.dk}\\
    \textbf{Semester Coordinator:}  Lene Tolstrup Sørensen\\
    \textbf{Secretary:}  Nele Pernille Staun Hvid
}{% the abstract
    There is a rising demand for smarter, and more sustainable campus environments as universities aim to optimize resources, while enhancing the educational quality of students.
    This report will look at leveraging technologies such as Bluetooth Low Energy to develop a tool that provides Aalborg University's administration and campus service with an overview of room utilization.
    The proposed solution involves deploying BLE-based indoor presence detection technology using a mesh network of ESP32 devices. This infrastructure will enable the collection and analysis of data related to the occupancy and usage patterns of indoor areas, providing actionable insights to inform decision-making. By harnessing BLE’s capabilities for reliable, low-power communication and combining it with a custom program written in C, the system aims to facilitate accurate tracking and statistical evaluation of room usage. Ultimately, this approach will contribute to more efficient resource management and a smarter, data-driven campus ecosystem.
}
% Here are the typical kinds of information found in most abstracts:

% 1. the context or background information for your research; the general topic under study; the specific topic of your research
% 2. the central questions or statement of the problem your research addresses
% 3. what’s already known about this question, what previous research has done or shown
% 4. the main reason(s), the exigency, the rationale, the goals for your research—Why is it important to address these questions? Are you, for example, examining a new topic? Why is that topic worth examining? Are you filling a gap in previous research? Applying new methods to take a fresh look at existing ideas or data? Resolving a dispute within the literature in your field? . . .
% 5. your research and/or analytical methods
% 6. your main findings, results, or arguments
% 7. the significance or implications of your findings or arguments.



% \cleardoublepage
% {\selectlanguage{danish}
% \pdfbookmark[0]{Danish title page}{label:titlepage_da}
% \aautitlepage{%
%   \danishprojectinfo{
%     Arduino BLE som tilstedeværelsessensor til statistisk analyse  %title
%   }{%
%     Optimering af pladsudnyttelse med BLE-sensorteknologi %theme
%   }{%
%     Efterårssemestret 2024 %project period
%   }{%
%     090 % project group
%   }{%
%     %list of group members
%     Alexander Bendix Pedersen \\
%     Alexander Hagelberg \\
%     Andrej Predrag Maricic \\
%     Janus Langkilde Pedersen \\
%     Jens Løvstrup Hvelplund Jaedeke \\
%     Johannes Dahl Rørdam, Mass Møller Vedel
%   }{%
%     %list of supervisors
%     Reza Tadayoni
%   }{%
%     1 % number of printed copies
%   }{%
%     \today % date of completion
%   }%
% }{%department and address
%   \textbf{Elektronik og IT}\\
%   Aalborg Universitet\\
%   \href{http://www.aau.dk}{http://www.aau.dk}
% }{% the abstract
%   Her er resuméet
% }}
